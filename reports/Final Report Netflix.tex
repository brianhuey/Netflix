\documentclass[11pt]{article}
\usepackage{verbatim}
\usepackage[hyphens]{url}
\usepackage{enumerate}
\setlength{\parindent}{0pt}
\setlength{\parskip}{10pt plus 6pt minus 4pt}
\usepackage{graphicx}
\usepackage{tikz}
\def\checkmark{\tikz\fill[scale=0.4](0,.35) -- (.25,0) -- (1,.7) -- (.25,.15) -- cycle;} 
\usetikzlibrary{shapes.geometric, arrows}

\addtolength{\oddsidemargin}{-.5in}
\addtolength{\evensidemargin}{-.5in}
\addtolength{\textwidth}{+1in}

\title{Final Report: Netflix!}
\author{ Brian Huey, Renee Rao }
\date{December 5th 2014}

\begin{document}
\maketitle

\section{Introduction}
We worked with the Netflix dataset used in the 2007 KDD cup competition which provides competitors with user ID, movie ID and timestamp for those who rated movies from the years 1998 to 2006. (http://www.kdd.org/kdd-cup-2007-consumer-recommendations)

We were asked to predict the probability that a user rated a given movie in 2006. To test this we were provided set of roughly 100\,000 (user\_id\, movie\_id) pairs where the user and movie IDs are drawn from the Netflix Prize training data set (where none of the pairs were rated in the training set.) Using this list we will try to predict the probability that each pair was rated in 2006 (i.e.the probability that user\_id rated movie\_id is in the 2006 set of ratings). It is important to note that the actual rating is irrelevant; we are only interested in whether the movie was rated by that user sometime in 2006. Our success at this task was computed by looking at the root mean squared error (RMSE) between our predictions for each pairs and actual values for the pairs (which are 0 or 1 if the pair is unrated or rated respectively). 

We were provided with a training data set of roughly 100.5 million ratings for the previous years of the format user\_id\,movie\_id\,date of rating. We note this task is very difficult, as a trivial method of predicting every movie is not rated gives an RMSE of 0.27. The KDD cup winner had an RMSE score of 0.256. \cite{1} We feel that an interesting feature is one that is different for rated movies than for unrated movies, so to assess feature utility heuristically, we graphed the distribution of each feature on the our validation set, and looked for features that produced a differential distribution for rated pairs versus unrated pairs. These graphs can be viewed and reproduced on Github. Using this metric we selected three features to input to our model.
   
\begin{enumerate}
\item Non-Negative Matrix Factorization (NMF) 
\item Global Effects of Time on User and Movie
\item Cosine Similarity
\end{enumerate}

We then computed these features for our validation set using AWS MapReduce and an EC2 instance, and used the distribution of these features to create a bayesian transformation used to normalize the features values on the test set, before feeding them into a logistic regression model to make predictions and compute RMSE. Our final result was an RMSE of .2667, which compares favorably with the basic prediction of all zeros but is significantly less than the winner of the 2007 competition. A detailed description of our data, features, transformation, model processing, and data selection follows. 

\section{The Data}

In this project we viewed the data pre-2005 as training data and 2005 as validation data and preserved the 2006 answer set as the test set for use in a final evaluation. This was done to avoid making predictions based on models or features that were calculated from data observed further in to the future than the validation data. We noted that the test set for 2006 was sampled proportionally to counts for user ratings and movie ratings made in 2006, so we created our 2005 validation sets accordingly, splitting them into:

\begin{itemize}
\item random subset of rated user,movie pairs in 2005   
\item popularity distributed random subset of unrated user, movie pairs from 2005 
\end{itemize}
	
Doing so allowed us to better analyze feature utility by visualizing the difference of features between the two sets as well as to compute the probability the was rated for given feature values used in the transformation step for the final 2006 test set. 

\section{Features}

\subsection{Non-negative matrix factorization (NNMF)} 
We used non-negative matrix factorization to develop user features and
movie features to then predict ratings for the Netflix data set. To do
this we estimated $A$, the matrix of movie and user ids, most of which
is considered unknown, by decomposing it into a user matrix, V and a
movie matrix, $U$ based on $K$ latent factors, such that $A \approx U
x V = \overline{A}$.  Under this model each row of matrix U is
considered a "movie factor"�� and each column of the matrix V is
considered a ``user factor''.  A prediction for a user-movie pair would
mean computing the dot product of the user factor vector and the movie
factor vector.

{\bf Algorithm Outline.}
Input: n by m matrix A, integer k
Output:  n by k matrix U, k by m matrix V with nonnegative entries.

{\bf Initialization}: Form initial matrix U (V) by choosing a random subset
of the columns (rows) of A and averaging them, K times.
We tune this so each element is expected to be added
to some U, one time. This is a parameter that
can be changed.

{\bf Main Loop}:
\begin{itemize}
\item {\bf Gradient Descent:}
We used gradient descent on the Mean Squared cost function for the difference
between $A$ and $\bar{A} = U \times V$.  We compute the RMSE in each
step

\begin{itemize}

\item summing MSE over non-zeros in A.
\item and then summing over random pairs of movie-users to 
ensure that $\bar{A}$ does not converge to all ones.

\end{itemize}

It would be prohibitive to sum over all non-zeros. 

\item
{\bf Nonnegativity:} We enforced positivity (which is certainly non-negative) 
on the weights in the matrix
factor by moving all weights away from zero if they
get too close to zero, and making negative weights
positive if the gradient pushes them to negative.

\item {\bf Spread:} We also normalized the factors (using a Gram-Schmidt type procedure to make sure all the factors don't simply repeat). We again enforced positivity here by staying away from zero.

\end{itemize}

The distribution of the NMF feature on validation ratings and non-ratings can be viewed in the nmf folder on Github.

\subsection{Global Effects}

The global effect feature is an adjustment to the overall probability that a user/movie pair is rated in 2006, assuming no information (i.e. number of ratings in the training set divided by the total number of user/movie pair combinations), which is approximately 1.17\%. Beginning at the first step, we started with the overall probability and subtracted the first effect from it, creating a residual which was the input to the next step, repeating until all effects have been accounted for. This process is based on work described by Bell and Koren.\cite{3}
\[global\_effect_{mu}=overall\_avg-avg_{m}-avg_{u}-time_{m}-time_{u}\]
Where \(u\) and \(m\) are user and movie, respectively.
\subsubsection{User/Movie Averages}
The user and movie averages were calculated for each unique user and movie id by taking the total number of observed ratings for that id and dividing it by the total number of possible ratings. For a given movie id, sum the total number of users rating that movie divided by the total number of users:
\[avg_{m} = \frac{\eta_{m}}{\sum_{u}x_{u}}\]
Where \(\eta_{m}\) is the number of ratings that movie m received and x is identically 1.
\subsubsection{Time Effects}
The time effect was estimated from the date timestamp for each row of data. For a given movie or user, we measure the time in days since the first rating and take the square root of the number of days. The coefficient, \(\hat{\theta}\) was estimated by the following equation provided by Bell and Koren:
\[\hat{\theta}_{m}=\frac{\sum_{u}r_{mu}x_{mu}}{\sum_{u}x_{mu}^{2}}\]
Where \(r_{ui}\) is the residual probability and \(x_{mu}\) is equal to the square root of the number of days since the movie was first rated.
The final estimate of \(\theta\) is:
\[\theta_{m}=\frac{\eta_{m}\hat{\theta}_{m}}{\eta_{m}+\alpha}\]
Where \(\alpha\) is a tuned parameter.
The distribution of the global effect feature on validation ratings and non-ratings can be viewed in the baseline folder on Github.

\subsection{Cosine Similarity}
For these features we view each movie as a vector with an entry for each user. The entry is a 1 if the movie was rated by the user and it is a zero otherwise. With this view we computed the cosine of the angle of the two movie vectors by taking their dot product and dividing it by the magnitudes of the two vectors. Movies with a high cosine similarity have a high fraction of users who have rated both movies. In this feature, for a user-movie pair, (U,M), we find all the movies rated by U and find their average value of the cosine similarity with M. The higher the value, the more likely we would expect this movie to be rated by this user, since many users rated this movie and other movies that this user also rated.

The distribution of the cosine similarity feature on validation ratings and non-ratings can be viewed in the cosine folder on Github.
 
\section{Transformation and Logistic Regression}
\subsection{Transformation}


Upon computing features for the test set we normalized the values of
each feature. The natural way to do this was to use Bayes rule to map
the arbitrary values of our features to probabilities of rating based
on those values. After using the pre-2005 training set to compute
features for our 2005 validation sets (described in the data section)
we used the distributions of the feature values on the rated
validation set and the unrated validation set to compute the
probability that a movie-user pair with a given feature value using
Bayes rule, that is to divide the number of pairs with that feature
value in the rated validation set by the total number of pairs with
that feature value in both validation sets (rated and unrated). We, of
course, must bucket our features values to do this computation.  We
then use this validation set mapping to transform features computed
for the 2006 test using the full training set into numbers which could
be viewed as 'probabilities'. We originally used these probabilities
for a naive Bayes model, but as some of our features appeared to be
correlated we decided to pass these normalized values into a logistic
regression model instead. This allows logistic regression to be fed
features that are between 0 and 1 and thus may avoid the danger of its
predictions being influenced too much by any single variation in a
feature.

\subsection{Logistic Regression}

We used logistic regression to make a model using data from rated and
unrated 2005 validation set pairs and the transformed features we
discussed above. The 'training set' for the model was balance among
rated and unrated pairs. The true test set, however is imbalance; the
rated pairs only comprise roughly 7 percent of the total.\cite{1} Thus
we scale the output of the logistic regression model so that the
average predicted probability is around 7 percent.

\section{Results and Discussion}

\subsection{Results}
Our results using the logistic regression model and combinations of our features are represented below, they should be compared to the 2007 KDD cup winner's RMSE of .256. (Note: The baseline RMSE is computed by predicting .07 for all pairs)

 \begin{center}
   \begin{tabular}{ | c | c | c | c | c |}
    \hline
    Baseline & Global & NMF & Cosine & RMSE \\ \hline
    \checkmark &  & & & .2684 \\ \hline
    \checkmark & \checkmark &  & & .2669 \\ \hline
     \checkmark & & \checkmark & & .2683\\ \hline
     \checkmark & & & \checkmark & .2668\\ \hline
      \checkmark & \checkmark & \checkmark & \checkmark & .2667\\ \hline
    \end{tabular}
\end{center}

\subsection{Discussion}

While our model did not match the performance of the first place
winner, its performance places it between the 3rd and 4th place groups
in the competition \cite{4}, so ultimately our model is learning
something.  Ideally we could perhaps have improved our performance by
running our features through a greater variety of models, tuning the
many parameters we used and adjusted heuristically when computing our
features, experimenting with validation methodology, and continuing to
explore other features such as outside information about release dates
of movies from websites like Rotten Tomatoes.  This process would be
more efficient now that we have built our pipeline for working with
these data and this task and pushed several features into it.



Please see Appendix for list of tools used and locations of data in S3.

\section{Division of Labor}
Note: Authorship, comments and readme's for all code can be found on the Github repo for the project. 

{\bf Brian:}

\begin{enumerate}
\item Join movie\_titles.txt to Rotten Tomatoes info 
\item Set up and organized github
\item Computed Global Effects feature using mapReduce
\item wrote an alternative baseline approach
\item organized and motivated data formatting
\end{enumerate}

{\bf Renee:}

\begin{enumerate}
\item Created Validation Sets. 
\item Wrote predict\_with\_features.py,  some\_tools.py, logistic.py to go from features computed on validation and test set to predictions and RMSE
\item Wrote Non-Negative Matrix Factorization Model and Cosine Similarity in Python.
\item Evaluated feature worthiness.
\item Generated associated figures for evaluation.
\end{enumerate}


\begin{thebibliography}{}

\bibitem{1}
M. Kurucz, A. A. Benczur, T. Kiss, I. Nagy, A. Szabo, and B. Torma. Who rated what: a combination of SVD, correlation and frequent sequence mining. In Proc. KDD Cup and Workshop, 2007. 
\bibitem{2}
J. Sueiras, A. Salafranca, and J. L. Florez, "A classical predictive modeling approach for Task Who rated what ? of the KDD CUP 2007," SIGKDD Explorations, vol. 9, no. 2, pp. 57-61, 2007. 
\bibitem{3}
Bell, R., Koren, Y. ?Improved Neighborhood-based Collaborative Filtering? KDD Cup ?07, August 12, 2007, San Jose, CA
\bibitem{4}
"KDD Cup 2007: Consumer Recommendations." Sig KDD. ACM SIGKDD, 12 Aug. 2007. Web. 05 Dec. 2014.

  \end{thebibliography}

\section{Appendix: Tools and S3 locations}

\subsection{Tools used}
\begin{itemize}
\item  sklean.liblinear.LogisticRegresion from the scikit package.
\item  A variety of python libraries including numpy, and scipy.spare. 
\item unix tools such as grep
\item python,  R 
\item AWS EC2, S3, MapReduce/Hadoop
\end{itemize} 

\subsubsection{S3 locations}
\begin{itemize}
\item User averages: s3://stat157-uq85def/home/sbhuey/outputs/user\_averages.txt
\item Movie averages: s3://stat157-uq85def/home/sbhuey/outputs/movie\_averages.txt
\item User coefficients: s3://stat157-uq85def/home/sbhuey/outputs/user\_coefficients.txt
\item Movie coefficients: s3://stat157-uq85def/home/sbhuey/outputs/movie\_coefficients.txt
\item Validation Sets: s3://stat157-uq85def/shared/netflix/feature\_samples
\item Test Set: s3://stat157-uq85def/home/reneerao/Data for Netflix project/KDD\_2007\_who\_rated\_what.fixed.txt
\item Validation sets with feature values appended s3://stat157-uq85def/home/reneerao/Data for Netflix project/validation\_sets\_features
\item Answer sets with feature values appended s3://stat157-uq85def/home/reneerao/Data for Netflix project/answer\_features
\item Pre-2005 training set: s3://stat157-uq85def/shared/netflix/pre2005\_training\_join/
\item Post-2005 validation set: s3://stat157-uq85def/shared/netflix/post2005\_training.txt
\item Training set: s3://stat157-uq85def/shared/netflix/training\_set/training\_set\_reshape.txt
\item Test set: s3://stat157-uq85def/shared/netflix/test\_sets/
\item Answer set: s3://stat157-uq85def/shared/netflix/answer\_sets/
\end{itemize}

\end{document}



