\documentclass[11pt]{article}
\usepackage{verbatim}
\usepackage[hyphens]{url}
\usepackage{enumerate}
\setlength{\parindent}{0pt}
\setlength{\parskip}{10pt plus 6pt minus 4pt}
\usepackage{graphicx}
\usepackage{tikz}
\def\checkmark{\tikz\fill[scale=0.4](0,.35) -- (.25,0) -- (1,.7) -- (.25,.15) -- cycle;} 
\usetikzlibrary{shapes.geometric, arrows}

\addtolength{\oddsidemargin}{-.5in}
\addtolength{\evensidemargin}{-.5in}
\addtolength{\textwidth}{+1in}

\title{Final Report: Netflix!}
\author{ Brian Huey, Renee Rao }
\date{December 5th 2014}

\begin{document}
\maketitle

\section{Introduction}
We are working the Netflix dataset used in the 2007 KDD cup competition which provides information on characteristics of users of the Netflix video services who rated movies from the years 1998 to 2006. (http://www.kdd.org/kdd-cup-2007-consumer-recommendations)

We need to predict which users rated which movies in 2006. To test this we use a provided set of roughly 100\,000 (user\_id\, movie\_id) pairs where the users and movies are drawn from the Netflix Prize training data set (where none of the pairs were rated in the training set.) Using this list we will try to predict the probability that each pair was rated in 2006 (i.e.the probability that user\_id rated movie\_id is in the 2006 set of ratings). It is important to note that the actual rating is irrelevant; we are only interested in whether the movie was rated by that user sometime in 2006. Our success at this task will be computed by looking at the root mean squared error (RMSE) between our predictions for each pairs and actual values for the pairs (which are 0 or 1 if the pair is unrated or rated respectively). 

We are provided with a data set of roughly 200 million ratings for the previous years of the format user\_id\,movie\_id\,date of rating.

We note this task is very difficult, as a trivial method of predicting every movie is not rated gives an RMSE of .27. The KDD cup winner had an RMSE score of .256.

We feel that an interesting feature is one that is different for rated movies than for unrated movies, so to assess feature utility heuristically, we graphed the distribution of each feature on the our validation set, and looked for features that produced a differential distribution for rated pairs versus unrated pairs. These graphs can be viewed below and reproduced on github. Using this metric we selected three features to input to our model.
   
\begin{enumerate}
\item Non-Negative Matrix Factorization (NMF) 
\item Global Effects of Time on User and Movie
\item Cosine Similarity
\end{enumerate}

We then computed these features for our validation set using AWS MapReduce and an EC2 instance, and used the distribution of these features to create a bayesian transformation used to normalize the features values on the test set, before feeding them into a logistic regression model to make predictions and compute RMSE. 

(PUT BEAUTIFUL GRAPHIC BELOW)
https://www.sharelatex.com/blog/2013/08/29/tikz-series-pt3.html

Our final result was an RMSE of .2667, which compares favorably with the basic prediction of all zeros but is significantly less than the winner of the 2007 competition. 

A detailed description of our features, transformation, model processing, and data selection follows. 

\section{The Data}

In this project we viewed the data pre-2005 as training data and 2005 as validation data and preserved the 2006 answer set as the test set for use in a final evaluation. We noted that the test set for 2006 was sampled proportionally to counts for user ratings and movie ratings made in 2006, so we created our 2005 validation sets accordingly, splitting them into:

\begin{itemize}
\item random subset of rated user,movie pairs in 2005   
\item popularity distributed random subset of unrated user, movie pairs from 2005 
\end{itemize}
	
Doing so allowed us to better analyze feature utility by visualizing the difference of features between the two sets as well as to compute the probability the was rated for given feature values used in the transformation step for the final 2006 test set. 

\section{Features}

\subsection{Non-negative matrix factorization (NNMF)} 
We use non-negative matrix factorization to develop user features and
movie features to then predict ratings for the Netflix data set. To do
this we estimate $A$, the matrix of movie and user ids, most of which
is considered unknown, by decomposing it into a user matrix, V and a
movie matrix, $U$ based on $K$ latent factors, such that $A \approx U
x V = \overline{A}$.  Under this model each row of matrix U is
considered a "movie factor"�� and each column of the matrix V is
considered a ``user factor''.  A prediction for a user-movie pair would
mean computing the dot product of the user factor vector and the movie
factor vector.

{\bf Algorithm Outline.}
Input: n by m matrix A, integer k
Output:  n by k matrix U, k by m matrix V with nonnegative entries.

{\bf Initialization}: Form initial matrix U (V) by choosing a random subset
of the columns (rows) of A and averaging them, K times.
We tune this so each element is expected to be added
to some U, one time. This is a parameter that
can be changed.

{\bf Main Loop}:
\begin{itemize}
\item {\bf Gradient Descent:}
We used gradient descent on the Mean Squared cost function for the difference
between $A$ and $\bar{A} = U \times V$.  We compute the RMSE in each
step

\begin{itemize}

\item summing MSE over non-zeros in A.
\item and then summing over random pairs of movie-users to 
ensure that $\bar{A}$ does not converge to all ones.

\end{itemize}

It would be prohibitive to sum over all non-zeros. 

\item
{\bf Nonnegativity:} We enforce positivity (which is certainly non-negative) 
on the weights in the matrix
factor by moving all weights away from zero if they
get too close to zero, and making negative weights
positive if the gradient pushes them to negative.

\item {\bf Spread:} We also normalize the factors (using a Gram-Schmidt type
         procedure to make sure all the factors don't simply repeat).
         We again enforce positivity here by staying away from zero.

\end{itemize}

The distribution of the NMF feature on validation ratings and non-ratings is below: 

\subsection{Global Effects}


The distribution of the Cosine Similarity feature on validation ratings and non-ratings is below: 

\subsection{Cosine Similarity}
For these feature we view each movie as a vector with an entry for each user. The entry is a 1 if the movie was rated by the user and it is a zero otherwise. With this view we can compute the cosine of the angle of the two movie vectors by taking their dot product and dividing it by the magnitudes of the two vectors. Initiatively movies with a high cosine similarity have a high fraction of users who have rated both movies. In this feature for a user-movie pair, (U,M), we find all the movies rated by U and find their average value of the cosine similarity with M. The high this value the more likely we would expect this movie to be rated by this user, since many users rated this movie and other movies that this user also rated.   

The distribution of the Cosine Similarity feature on validation ratings and non-ratings is below: 

\section{Transformation and Logistic Regression}

\subsection{Transformation}
Upon computing features for the test set we normalized  the values of each feature. The natural way to do this was to use Bayes rule to map the arbitrary values of our features to probabilities of rating based on those values. After using the pre-2004 training set to compute features for our 2005 validation sets (described in the data section) we used the distributions of the feature values on the rated validation set and the unrated validation set to compute the probability that a movie-user pair with a given feature value using Bayes rule, that is to divide the number of pairs with that feature value in the rated validation set by the total number of pairs with that feature value in both validation sets (rated and unrated). We, of course, must bucket our features values to do this computation.  We then use this validation set mapping to transform features computed for the 2006 test using the full training set into numbers which could be viewed as 'probabilities'. We originally used these probabilities for a naive Bayes model, but as some of our features appeared to be correlated we decided to pass these normalized values into a logistic regression model instead. This allows logistic regression to be fed features that are between 0 and 1 and thus may avoid the danger of its predictions being influenced too much by any single variation in a feature.  

\subsection{Logistic Regression}
We used logistic regression to make a model using data from rating and unrated 2005 validation set pairs and the transformed features we discussed above. The 'training set' for the model was balance among rated and unrated pairs. The true test set, however is imbalance; the rated pairs only comprise roughly 7 percent of the total. Thus we scale the output of the logistic regression model so that the average predicted probability is around 7 percent. 

\section{Results and Discussion}

\subsection{Results}
Our results using the logistic regression model and combinations of our features are represented below, they should be compared to the 2007 KDD cup winner's RMSE of .256. (Note: The baseline RMSE is computed by predicting .07 for all pairs)

 \begin{center}
   \begin{tabular}{ | c | c | c | c | c |}
    \hline
    Baseline & Global & NMF & Cosine & RMSE \\ \hline
    \checkmark &  & & & .2684 \\ \hline
    \checkmark & \checkmark &  & & .2681 \\ \hline
     \checkmark & & \checkmark & & .2682\\ \hline
     \checkmark & & & \checkmark & .2682\\ \hline
      \checkmark & \checkmark & \checkmark & \checkmark & .2667\\ \hline
    \end{tabular}
\end{center}

\subsection{Discussion}
While our model could not compete with the first place winner it did place us between the 3rd and 4th place groups in the competition, so ultimately our model is learning something. Ideally we would like to further this project by running our features through a greater variety of models, tuning the many parameters we used and adjusted heuristically when computing our features, and continuing to explore other features that demonstrated promise on the validation sets, such as outside information about release dates of movies from websites like Rotten Tomatoes. Ultimately, however, we did not have enough time to flesh the project out as fully as we originally envisioned, as we spent quite a bit of time computing our current features and testing them on our validation sets. Creating the validation set correctly was particularly key to getting good measure of the utility of our features, and ultimately cost us some time. 


\section{Tools}
\begin{itemize}
\item  sklean.liblinear.LogisticRegresion from the scikit package.
\item  A variety of python libraries including numpy, and scipy.spare. 
\item unix tools such as grep
\item python,  R 
\item AWS EC2, S3, MapReduce/Hadoop
\end{itemize}
\section{Division of Labor}
Note: Authorship, comments and readme's for all code can be found on the github website for the project. 
{\bf Brian}

\begin{enumerate}
\item Join movie\_titles.txt to Rotten Tomatoes info 
\item Set up and organized github
\item Upload of data to sc3
\item Calculate the average number of movies rated over all users in the training set (mapreduce).
\item Calculate the average number of ratings over all movies in the training set (mapreduce).
\item  Calculate the average number of movies rated for each user in the training set (mapreduce).
\item  Calculate the average number of ratings for each movie in the training set (mapreduce).
\item Researched K Nearest Neighbor algorithms.
\item Alternate Baseline method.
\end{enumerate}

{\bf Renee:}

\begin{enumerate}
\item Created Validation Sets. 
\item Wrote predict_with_features.py,  some_tools.py, logistic.py to go from features computed on validation and test set to predictions and RMSE
\item Wrote Non-Negative Matrix Factorization Model and Cosine Similarity in Python.
\item Evaluated feature worthiness.
\item Generated associated figures for evaluation.
\item Wrote reports. 
\end{enumerate}







\end{document}



